\documentclass{article}
\usepackage[utf8]{inputenc}
\usepackage[T1]{fontenc} 
\usepackage[french]{babel}
\usepackage{graphicx}
\usepackage[table]{xcolor}
\usepackage{geometry}
\usepackage{hyperref}
\usepackage{appendix}

\geometry{hmargin=3cm,vmargin=3.5cm}

\title{Rapport de Stage}
\author{Laureline MARTIN}
\date{Mercredi 7 Octobre 2020}

\begin{document}
\maketitle

\newpage
\renewcommand{\contentsname}{Table des matières}\tableofcontents

\newpage
\section{Introduction}
	Dans le cadre de la validation du Master 2 Data Managment in a Digital Word - Datascale, proposé par l'Université de Versailles Saint-Quentin, j'ai effectué un stage de 6 mois au sein d'un laboratoire de recherche.
	J'ai poursuivi ce stage en majorité en télétravail, dû à la crise sanitaire de la COVID-19.\par
	Dans ce rapport, je vais tout d'abord présenter le laboratoire ainsi que l'équipe avec laquelle j'ai travaillé. Puis je vais présenter ma mission principale et les tâches que j'ai effectuée.

\section{Contexte du stage}
	\subsection{Le laboratoire CEDRIC}
		Pour ce stage, j'ai été accueillie par le laboratoire de Centre d'Etudes et De Recherche en Informatique et Communication (CEDRIC)\footnote{\href{http://cedric.cnam.fr/lab/accueil/labo/}{http://cedric.cnam.fr/lab/accueil/labo/}} au sein du Conservatoire National des Arts et Métiers (CNAM) de Paris. 
		Le laboratoire se situe au 2 rue Conte 75003 PARIS.\par
		Le laboratoire CEDRIC est composé de huit équipes ayant des actions dans le domaine de l'informatique fondamentale et appliquée, ainsi que dans d'autres disciplines proches (statistiques, électronique,...).
		\begin{itemize}
			\item L'équipe Données complexes, apprentissage et représentations (Vertigo) : Extraire de l'information et construire des méthodes de gestion de données basées sur le contenu pour des données massives audios, images et vidéos;
			\item L'équipe Interactivité pour Lire et Jouer (ILJ) : Questions autour de l'interactions homme-machine concernant des activités telles que le jeu et la lecteure. Cette équipe mèle plusieurs disciplines (informatiques, psychologie, design, arts,...) afin de répondre aux études en cours qui concernent la modélisation de la difficulté dans les jeux, les méthodeologies de game design inclusif,...;
			\item L'équipe Ingénieurie des Systèmes d'Information et de Communication (ISID) : Réunie autour de trois axes de recherche (les systèmes décisionnels, le web sémantique et la qualité des systèmes d'information) dans le but de concevoir des méthodes, des outils et des techniques	pour la conception et l'analyse de systèmes d'information et de décision dans tous les domaines;
			\item L'équipe Traitement du signal et architectures électroniques (LAETITIA) : Concentrée sur trois axes de recherhe : traitement du signal pour les télécommunication (recherches en lien avec les réseaux cellulaires (5G/6G) et en lien avec les problématiques de couches physique pour les réseaux faible puissance longue portée pour l'IoT), sûreté de fonctionnement des sytèmes dynémiques (recherche en automatique) et implémentation temps-réel (mise en oeuvre des algorithmes proposés par l'équipe) ;
			\item L'équipe Méthode Statistique de Data-Mining et Apprentissage (MSDMA) : Traitement de données et développement de modles d'apprentissage statistiques dans le but d'extraire des information pour prise de décision;
			\item L'équipe Optimisation Combinatoire (OC) : Rassemblée autour de deux axes : la programmation mathématique et applications ainsi que les graphes et optimisation;
			\item L'équipe Réseaux et Objets Connectés (ROC) : Porte sur l'analyse et l'exploitation des nouvelles architectures réseaux et des systèmes liés à la virtualisation, à la mobilité et au développemnt des objets connectés;
			\item L'équipe Systèmes Sûrs (SYS) : Spécialisation, conception, vérification et évaluation des systèmes. Pour cela, Les recherches se font autour de trois axes : l'axe typage, sémantique et preuve, l'axe architecture logiciel, architecture systèmes et ligne de produit et l'axe vérification  et évaluation de systèmes parallèle et asynchrone.
		\end{itemize}
	\subsection{L'équipe et son projet}
		J'ai été encadrée par 
		\begin{itemize}
			\item Mme Elena KORNYSHOVA, Maître de conférences travaillant au sein de l'équipe \textit{Igénieurie des systèmes d'information et de décision}, 
			\item Mme Viviane GAL, ingénieure travaillant au sein de l'équipe \textit{Interactivité pour lire et jouer},
			\item M Eric GRESSIER-SOUDAN, professeur des universités travaillant au sein de l'équipe \textit{Réseaux et Objets Connectés}
		\end{itemize}
		Cette équipe porte un projet exploratoire du CEDRIC. 
		Le projet "Conception et développement des jeux pervasifs adaptables avec la prise en compte des états émotionnels des joueurs".
		L'objectif gloabl est de proposer une expérience très immersive aux joueurs. Pour cela, le jeu doit pouvoir s'adapter au joueur en fonction du contexte global.
		Dans un premier temps, le but de ce projet est de pouvoir formuler un modèle conceptuel du jeu pervasif adaptatif basé sur les émotion et sur le contexte. Et dans un second temps, de pouvoir développer une approche d'ingénerie situationnelle du système d'information pour ce type de jeu.\newline
		Mon stage s'inscrit au sein de cette équipe et pour ce projet 

\section{Objectif du stage}
	Le but principal de mon stage a été de pouvoir réaliser une ontologie sous forme de diagramme de classe. 
	Cette ontologie devait pouvoir mettre en évidence les briques principale pour élaborer un jeu pervasif utilisant un modèle utilisateur capable de prendre en charge les émotions du joueurs et d'autres traits liés à lui.\newline
	Pour ce faire, il a fallut tout d'abord faire l'état de l'art de ce qui existait dans la littérature scientifique.

\section{Etat de l'art (familiariation avec le projet)}
	Le premier jour de mon stage, mes tuteurs m'ont remis 5 documents qui m'ont servi de base pour entrer dans le sujet du jeu pervasif (avec \cite{gal_2019}), la reconnaissance des émotions et des états émotionnels (avec \cite{gal_2019,gizycka_et_al._2018,nalepa_et_al._2019}), le modèle utilisateur (avec \cite{alhudar_2019}) et les Decision Making avec (avec \cite{kornyshova_et_al._2010}).\newline
	Aprèss la lecture de ces documents, 

\section{Approche suivie et solution proposée}
	\subsection{Rédaction d'un état de l'art}
	\subsection{Modélisation d'une ontologie}
	\subsection{Prototype avec l'environnement Kafka}

\section{Validation}

\section{Conclusion}


\bibliographystyle{abbrv}
\bibliography{include/biblio}
\end{document}