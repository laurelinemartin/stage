\documentclass{article}
\usepackage[utf8]{inputenc}
\usepackage[T1]{fontenc} 
\usepackage[french]{babel}
\usepackage{graphicx}
\usepackage[table]{xcolor}
\usepackage{geometry}
\usepackage{hyperref}
\usepackage{appendix}

\geometry{hmargin=3cm,vmargin=3.5cm}

\title{Rapport de Stage}
\author{Laureline maketitlertin}
\date{Mercredi 7 Octobre 2020}

\begin{document}
\maketitle
\renewcommand{\contentsname}{Table des matières}\tableofcontents
\newpage
\section{Introduction}
\section{Contexte du stage}
	Pour ce stage, j'ai rejoins le laboratoire CEDRIC au sein du Conservatoire National des Arts et Métiers de Paris. 
	Le laboratoire se situe au 2 rue Conte 75003 PARIS.\newline
	Les mission principales du laboratoire sont :
	\begin{itemize}
		\item
	\end{itemize}
	Et le futurs projets que le laboratoire souhaite mener:
	\begin{itemize}
		\item
	\end{itemize}
	On retrouve les missions du laboratoire CEDRIC sur : \href{}{}.\newline
\section{Objectif du stage}
	Le but principal de mon stage a été de pouvoir réaliser une ontologie sous forme de diagramme de classe. 
	Cette ontologie devait pouvoir mettre en évidence les briques principale pour élaborer un jeu pervasif utilisant un modèle utilisateur capable de prendre en charge les émotions du joueurs et d'autres traits liés à lui.\newline
	Pour ce faire, il a fallut tout d'abord faire l'état de l'art de ce qui existait dans la littérature scientifique.
\section{Etat de l'art}
	Le premier jour de mon stage, mes tuteurs m'ont remis 5 documents qui m'ont servi de base pour entrer dans sujet du jeu pervasif (avec \cite{gal_2019}), la reconnaissance des émotions et des états émotionnels (avec \cite{gal_2019,gyzicka_et_al_2018,nalepa_et_al_2019}), le modèle utilisateur (avec \cite{\alhudar_2019}) et les Decision Making avec (\cite{Kornyshova_et_al_2013}).\newline
	Ar_s la lecture de ces
\section{Approche suivie et solution proposée}
\section{Validation}
\section{Conclusion}

\end{document}