\documentclass{article}
\usepackage[utf8]{inputenc}
\usepackage[T1]{fontenc} 
\usepackage[french]{babel}
\usepackage{graphicx}
\usepackage{subfigure}
\usepackage[table]{xcolor}
\usepackage{geometry}
\usepackage{hyperref}
\usepackage{appendix}
\usepackage{pdfpages}

\geometry{hmargin=2.5cm,vmargin=3cm}

\title{Rapport de Stage}
\author{Laureline MARTIN}
\date{Mercredi 7 Octobre 2020}

\begin{document}
\maketitle

\newpage
\renewcommand{\contentsname}{Table des matières}\tableofcontents

\newpage
\section{Résumé}
	Dans le cadre de la validation du Master 2 Data Managment in a Digital World - Datascale, proposé par l'Université de Versailles Saint-Quentin / Paris Saclay, j'ai effectué un stage de cinq mois et demi au sein du laboratoire de recherche CEDRIC du Conservatoire National des Arts et Métiers de Paris.
	J'ai poursuivi ce stage en majorité en télétravail, dû à la crise sanitaire de la COVID-19.

\section{Abstract}
	In order to validate my Master 2 Data Management in a Digital World - Datasclae by Université de Versailles Saint-Quentin / Paris Saclay, I worked for the CEDRIC research laboratory based on Conservatoire Nationnal des Arts et Métiers at Paris during five mounths and half.
	Due to COVID-19 pandemic, I mostly worked at home.

\section{Introduction}
	Le jeu pervasif est un type de jeu très immersif. 
	Il mêle le monde physique et le monde numérique et rend floue cette frontière. 
	Le projet exploratoire initié par trois enseignants-chercheurs du laboratoire CEDRIC intitulé "Conception et développement des jeux pervasifs adaptables avec la prise en compte des états émotionnels des joueurs" vise à élaborer un jeu pervasif capable de s'adapter dynamiquement selon le contexte du joueur et son environnement. 
	Cependant cette conception doit rester assez générique pour pouvoir l'appliquer à d'autres domaines dans de prochains projets.\par
	J'ai rejoins ce projet à son commencement. 
	Le but de mon stage est double : 
	d'une part, je dois recueillir et synthétiser les éléments déjà existants dans la littérature scientifique et industrielle pour l'élaboration d'un tel jeu pervasif. 
	Et d'autre part, je dois élaborer un modèle conceptuel afin de représenter les éléments qui permettront par la suite d'implémenter ce jeu particularisé.\par
	Dans ce rapport, je vais tout d'abord présenter le laboratoire ainsi que l'équipe avec laquelle j'ai travaillé. Puis je présenterai ma mission principale et les tâches que j'ai effectué.
	Pour terminer, j'aborderai la validation des tâches menées.

\section{Contexte du stage}
	\subsection{Le laboratoire CEDRIC}
		\centering \includegraphics[scale=0.35]{include/logo-cedric.PNG}\\
		Pour ce stage, j'ai été accueillie par le laboratoire de Centre d'Etudes et De Recherche en Informatique et Communication (CEDRIC)\footnote{\href{http://cedric.cnam.fr/lab/accueil/labo/}{http://cedric.cnam.fr/lab/accueil/labo/}} au sein du Conservatoire National des Arts et Métiers (CNAM) de Paris\footnote{\href{http://www.cnam.fr/portail/accueil-conservatoire-national-des-arts-et-metiers-821166.kjsp}{http://www.cnam.fr/portail/accueil-conservatoire-national-des-arts-et-metiers-821166.kjsp}}. 
		Le laboratoire se situe au 2 rue Conte 75003 PARIS.\par
		Le laboratoire CEDRIC est composé de huit équipes ayant des actions dans le domaine de l'informatique fondamentale et appliquée, ainsi que dans d'autres disciplines proches (statistiques, électronique,...).
		\begin{itemize}
			\item L'équipe Données complexes, apprentissage et représentations (Vertigo) : Extraire de l'information et construire des méthodes de gestion de données basées sur le contenu pour des données massives audios, images et vidéos;
			\item L'équipe Interactivité pour Lire et Jouer (ILJ) : Questions autour de l'interaction homme-machine concernant des activités telles que le jeu et la lecteure. Cette équipe mèle plusieurs disciplines (informatiques, psychologie, design, arts,...) afin de répondre aux études en cours qui concernent la modélisation de la difficulté dans les jeux, les méthodologies de game design inclusif,...;
			\item L'équipe Ingénieries des Systèmes d'Information et de Décision (ISID) : Réunie autour de trois axes de recherche (les systèmes décisionnels, le web sémantique et la qualité des systèmes d'information) dans le but de concevoir des méthodes, des outils et des techniques	pour la conception et l'analyse de systèmes d'information et de décision dans tous les domaines;
			\item L'équipe Traitement du signal et architectures électroniques (LAETITIA) : Concentrée sur trois axes de recherhe : traitement du signal pour les télécommunications (recherches en lien avec les réseaux cellulaires (5G/6G) et en lien avec les problématiques de couches physique pour les réseaux faible puissance longue portée pour l'IoT), sûreté de fonctionnement des sytèmes dynémiques (recherche en automatique) et implémentation temps-réel (mise en oeuvre des algorithmes proposés par l'équipe);
			\item L'équipe Méthode Statistique de Data-Mining et Apprentissage (MSDMA) : Traitement de données et développement de modèles d'apprentissage statistiques dans le but d'extraire des information pour prise de décision;
			\item L'équipe Optimisation Combinatoire (OC) : Rassemblée autour de deux axes : la programmation mathématique et applications ainsi que les graphes et optimisation;
			\item L'équipe Réseaux et Objets Connectés (ROC) : Porte sur l'analyse et l'exploitation des nouvelles architectures réseaux et des systèmes liés à la virtualisation, à la mobilité et au développemnt des objets connectés;
			\item L'équipe Systèmes Sûrs (SYS) : Spécialisation, conception, vérification et évaluation des systèmes. Pour cela, Les recherches se font autour de trois axes : l'axe typage, sémantique et preuve, l'axe architecture logiciel, architecture systèmes et ligne de produit et l'axe vérification  et évaluation de systèmes parallèles et asynchrones.
		\end{itemize}
	\subsection{L'équipe et son projet}
		J'ai été encadrée par 
		\begin{itemize}
			\item Mme Elena KORNYSHOVA, Maître de conférences travaillant au sein de l'équipe \textit{Igénieurie des systèmes d'information et de décision}, 
			\item Mme Viviane GAL, ingénieure travaillant au sein de l'équipe \textit{Interactivité pour lire et jouer},
			\item M Eric GRESSIER-SOUDAN, professeur des universités travaillant au sein de l'équipe \textit{Réseaux et Objets Connectés}
		\end{itemize}
		Cette équipe porte un projet exploratoire du CEDRIC : le projet "Conception et développement des jeux pervasifs adaptables avec la prise en compte des états émotionnels des joueurs".\par
		L'objectif gloabl de ce projet est de proposer une expérience très immersive aux joueurs. 
		Pour cela, le jeu doit pouvoir s'adapter au joueur en fonction du contexte global.
		Dans un premier temps, le but de ce projet est de pouvoir formuler un modèle conceptuel du jeu pervasif adaptatif basé sur les émotions et sur le contexte. 
		Et dans un second temps, de pouvoir développer une approche d'ingénerie situationnelle du système d'information pour ce type de jeu.\newline
		Mon stage s'inscrit au sein de cette équipe et pour ce projet 

\section{Objectif du stage}
	J'ai intégré le projet "Conception et développement des jeux pervasifs adaptables avec la prise en compte des états émotionnels des joueurs" à son commencement. 
	Ma mission principale a été d'élaborer un modèle conceptuel pour représentant un jeu pervasif adaptable aux émotions du joueur.
	Ce modèle devait pouvoir mettre en évidence les briques principales pour élaborer un jeu pervasif utilisant un modèle utilisateur capable de prendre en charge les émotions du joueur et d'autres traits liés à lui et à son environnement (voir l'Offre de stage Annexe \ref{app:annexe1}).
	% Pour ce faire, il a fallut tout d'abord faire l'état de l'art de ce qui existait dans la littérature scientifique. 

\section{Approche suivie et solution proposée}
	\subsection{Recherches bibliographiques}
		Le premier jour de mon stage, mes tuteurs m'ont remis 5 documents qui m'ont servi de base pour entrer dans le sujet du jeu pervasif avec \cite{gal_2019}, la reconnaissance des émotions et des états émotionnels avec \cite{gal_2019,gizycka_et_al._2018,nalepa_et_al._2019}, le modèle utilisateur avec \cite{alhudar_2019} et les Decision Making avec \cite{kornyshova_et_al._2010}.\par
		Les deux premières semaines, j'ai fait la lecture de ces documents et la présentation des approches de chacun de ces documents à mes tuteurs.
		Après ces deux semaines, j'ai recherché de nouveaux documents ressources autour de la reconnaissance des émotions dans les jeux comme me l'ont suggérés mes tuteurs. 
		Pour plus de facilité à centraliser mes recherches, j'ai utilisé l'outil de gestion de ressources Zotero\footnote{\href{https://www.zotero.org}{https://www.zotero.org}}. 
		Pour mes recherches, j'ai utilisé en grande majorité Google Scholar. 
		Etant à domicile pour mon stage, je n'avais pas accès à une bibliothèque particulière que j'aurais pu avoir au CNAM. 
		Cependant, j'ai eu la chance de trouver la majorité des références qui m'ont été utiles par la suite librement sur internet. 
		Pour les quelques références que je n'ai pas pu trouver gratuiteemnt, je me suis tournée vers mes tuteurs qui ont pu me les fournir.\par
		Pedant plusieurs semaines, j'ai présenté les approches que je trouvais à mes tuteurs lors de réunions hébdomadaire via Skype.\par
		Au fil de mes recherches, j'ai construit un tableau synthétique des approches (figures \ref{fig:tabsynt1} et \ref{fig:tabsynt2}) en utilisant plusieurs colonnes-critères afin que la lecture et l'explication de chacune d'entre elle soit plus simple.\par
		\begin{figure}
			\includegraphics[scale=0.47]{include/tri1.PNG}\\
			\includegraphics[scale=0.47]{include/tri2.PNG}\\
			\includegraphics[scale=0.47]{include/tri3.PNG}\\
			\includegraphics[scale=0.47]{include/tri4.PNG}
			\caption{Tableau synthétique des références (première partie)}
			\label{fig:tabsynt1}
		\end{figure}
		\begin{figure}
			\includegraphics[scale=0.47]{include/tri5.PNG}\\
			\includegraphics[scale=0.47]{include/tri6.PNG}\\
			\includegraphics[scale=0.47]{include/tri7.PNG}\\
			\includegraphics[scale=0.47]{include/tri8.PNG}
			\caption{Tableau synthétique des références (seconde partie)}
			\label{fig:tabsynt2}
		\end{figure}
		Ce tableau, que j'ai rempli et étoffé pendant mes recherches bibliographiques, m'a aidé à mieux comprendre les approches en catégorisant les méthodes et les expérimentations qui étaient décritent dans les références. 
		Ce tableau m'a également beaucoup aidé à transmettre les idées globales lors des réunions.\par
		A partir de fin avril, mes tuteurs m'ont chargé de nouvelles missions. J'ai travaillé sur la rédaction d'un état de l'art et sur l'élaboration d'un modèle sous forme de diagramme de classe en parallèle.
	\subsection{Rédaction d'un état de l'art}
		Pour la rédaction de l'état de l'art, j'ai dans un premier temps réfléchi à la problématique et à l'objectif que je voulais décrire. Mes tuteurs m'ont suggéré de commencer par écrire une première version de l'introduction. Cela m'a permi de réfléchir à la structure et aux idées que je voulais transmettre.
		Après quelques corrections de leur part, j'ai pu recentrer mes propos et mieux définir la direction que je voulais pour mon état de l'art.\par
		J'ai ensuite défini un plan que j'ai détaillé avec les points qui me semblait important à développer dans chaque partie. Je suis revenue plusieurs fois sur ce plan avec les correction et indications de mes tuteurs.\par
		Une fois que le plan m'a semblé pertinent et que l'introduction me convenait, j'ai développé les parties. Mes tuteurs m'ont fait plusieurs retours, me donnant des indications sur la forme de l'étt de l'art. 
		J'ai compris que l'état de l'art que je rédigeais devait suivre des règles précises. 
		Il devait dès la table des matières exprimer l'aspect descriptif et l'aspect comparatif.\par
		J'ai aussi compris qu'il fallait que je sois très précise sur mes critères de comparaison. 
		Lorsque j'ai commencé à rédiger, j'ai choisi trois critères pour comparer les approches. 
		En réalité, ces critères de comparaison ressemblaient plus à des "super-critères". 
		Pour corriger ce problème, j'ai repris les idées qui étaient dans le tableau synthétique que j'avais créé au début de mon stage (voir figure \ref{fig:tabsynt1} et \ref{fig:tabsynt2}). 
		Je me suis inspirée des colones de ce tableau pour faire des critères plus ciblés. 
		Ce tableau m'aura aussi aidé à rédiger le tableau comparatif de l'état de l'art.\par
		L'état de l'art se trouve en annexe \ref{ann:eda}.
	\subsection{Conception du modèle}
		En parallèle de l'état de l'art, j'ai travaillé sur la conception d'un modèle conceptuel. 
		Le but de ce modèle était de représenter chaque composant permettant la prise en compte des états émotionnels dans le jeu pervasif. 
		Il fallait à la fois représenter le joueur, le jeu, les états émotionnels, les capteurs mais aussi des composants plus abstraits comme les réactions physiologiques du joueur, la reconnaissance des états émotionnels, l'adaptation dynamique du jeu, etc.\par
		Dès le début de cette conception, nous avons choisi de représenter ce modèle à l'aide d'un diagramme de classe UML. 
		Etant donné que j'avais déjà appris à travailler avec ce type de diagramme durant mon cursus, j'ai pu directement commencer à travailler sur le modèle, sans avoir à me documenter sur les bases du diagramme de classe.\par
		D'un côté, j'ai travaillé au brouillon sur papier afin de mettre mes idées en place.
		Et de l'autre côté, j'ai travaillé sur ordinateur afin de mettre au propre les brouillons qui me convenaient.
		Pour cela, j'ai utilisé l'éditeur de diagrammes en ligne Visual-Paradigm\footnote{\href{https://online.visual-paradigm.com/fr/}{https://online.visual-paradigm.com/fr/}}.\par
		Pour la conception de ce modèle, j'ai fais un premier modèle très peu détaillé (voir figure \ref{fig:premmodele}, sans attribut ni opération.
		\begin{figure}
			\subfigure[Première version]{\includegraphics[scale=0.49]{include/modele-v1-1.PNG}}\label{fig:vers1}\\
			\subfigure[Deuxième version]{\includegraphics[scale=0.46]{include/modele-v1-2.PNG}}\label{fig:vers2}
			\caption{Les premières versions de la conception d'un modèle représentant un jeu pervasif prenant en compte l'état émotionel du joueur}
			\label{fig:premmodele}
		\end{figure}
		Le but était d'avoir une première idée de ce à quoi pourrait ressembler ce modèle et transmettre mes idées à mes tuteurs.\par
		Après plusieurs réunions pour réajuster les classes du modèle, j'ai commencé à "entrer" dans chaque classe pour y ajouter les attributs et les opérations. 
		En faisant cela, j'ai pu me rendre compte que certaines classes ne prenaient pas d'attributs ou ne prenaient pas d'opération.
		Cela m'a permis de me rendre compte que ces classes pouvaient être intégrées dans d'autres classes.
		Cela permettait une meilleure lisibilité du modèle.\par
		La classe  "\texttt{Elements Externes}" est un peu à part. 
		Je ne l'ai jamais "développée" mais il était important de la représenter.
		Effectivement, cette classe permet de représenter l'ensemble du contexte autour du joueur et du jeu qui doit être pris en compte dans de futurs étapes, mais qui ne concernent pas notre modèle.\par
		Après plusieurs semaines, le modèle est représenté par la figure \ref{fig:modele}.
		\begin{figure}
			\includegraphics[scale=0.5]{include/modele-v2-5.png}
			\caption{Le jeu pervasif prenant en compte l'état émotionnel du joueur}
			\label{fig:modele}
		\end{figure}
		Pour mieux comprendre ce modèle et les classes qui le composent j'ai rédigé un fichier qui détaille chaque classe une à une (voir annexe \ref{ann:detailclasse}).\par
		J'ai également fait une version "exemple" du diagramme où chaque attribut à une valeur (voir figure %\ref{fig:exmodele}
		). \textit{ajouter l'exemple}.
		Cette version du diagramme permet d'avoir une vision plus concrète.\par
		Par la suite, j'ai conçu un diagramme par package (voir figure \ref{fig:diagpack}.
		Cela permet de se représenter autrement les différents éléments mis en jeu dans ce projet.
		Le diagramme par package permet aussi de visualiser, très globalement, l'implémentation qui pourrait être possible par la suite.
		\begin{figure}
			\centering
			\includegraphics[scale=0.3]{include/diagrammePackage.png}
			\caption{Diagramme par package d'un jeu pervasif prenant en compte l'état émotionnel du joueur}
			\label{fig:diagpack}
		\end{figure}
	\subsection{Prototype avec l'environnement Kafka}
		\textit{Choix entre RabbitMQ et Kafka.\newline
		Mettre la synthèse.\newline
		Apprendre à utiliser Kafka ?}

\section{Validation}
	\textit{que mettre ici ???}

\section{Conclusion}
	\textit{faire la conclusion.}


\newpage
\appendix
\section{Annexe 1 : Offre de stage}\label{app:annexe1}
	\textbf{Elaboration du modèle conceptuel des jeux pervasifs adaptables avec la prise en compte des états émotionnels des joueurs}\par
	\medskip
	\textbf{Contexte :}\newline
	Le champ des jeux affectifs est nouveau. Il s’appuie sur l’intégration de nouveaux moyens à développer dans les jeux afin d’adaptabilité. [1] et [2] présentent une méthodologie unifiée pour la conception des jeux affectifs utilisant le plus tôt possible le mécanisme de boucle émotionnelle. Ils repèrent des variations à l’aide de mesures physiologiques et appliquent un modèle issu d’un ensemble construit considéré comme en relation avec les émotions. Leur étude montre combien la dimension émotionnelle de l’utilisateur est importante mais difficile à gérer.\newline
	Le profil du joueur, y compris ses émotions, impacte la conception des jeux. Afin de proposer une meilleure expérience aux joueurs et de proposer un jeu particularisé, le jeu doit être adaptable en fonction du contexte global du joueur. Nous sommes dans une approche holistique qui combine à la fois l’individu et ses émotions, et, les influences de l’entourage qui va du bâtiment lui-même à l’atmosphère que dégage le lieu. Très peu de travaux ont été faits pour la conception et le développement des jeux adaptables dynamiquement. [3] formalise le concept des jeux appliqués aux visites de musées. Ce travail modélise le jeu de visite et propose un processus d’équilibrage entre la dimension ludique et la dimension non ludique (la visite) de ce type de jeux. [3] propose des patrons de mission qui servent d’éléments réutilisables lors de la conception des jeux, mais qui ne couvrent qu’une partie du processus de conception.\par
	\textbf{Sujet :}\newline
	Il s’agit dans ce stage d’élaborer un modèle conceptuel du jeu pervasif adaptable basé sur les émotions. Ce modèle, éventuellement réalisé sous forme d’une ontologie, doit couvrir toute la variété des facteurs qui impactent le jeu tels que le profil de l’utilisateur et ses données physiologiques exprimant son état émotionnel. Cette ontologie doit être construite de façon à ce qu’elle soit adaptée à la démarche situationnelle nécessaire pour la composition dynamique du jeu.

\section{Annexe 2 - Adaptation d’un jeu pervasif particularisé basée sur l'état émotionnel et sur les caractéristiques du joueur – État de l’art}\label{ann:eda}
	\includepdf[pages=-]{include/eda.pdf}

\section{Annexe 3 - Les classes en détail}\label{ann:detailclasse}
	%\includepdf[pages=-]{include/classes.pdf}

\section{Annexe 4 - Synthèse comparative entre Kafka et RabbitMQ}%\label{ann:kafkarabbitmq}
	\includepdf[pages=-]{include/comparatifKafkaVSRabbitMQ.pdf}	

\bibliographystyle{abbrv}
\bibliography{include/biblio}
\end{document}