\documentclass{article}
\usepackage[utf8]{inputenc}
\usepackage[T1]{fontenc} 
\usepackage[french]{babel}
\usepackage{graphicx}
\usepackage[table]{xcolor}
\usepackage{geometry}
\usepackage{hyperref}
\usepackage{appendix}

\geometry{hmargin=3cm,vmargin=3.5cm}

\title{Rapport de Stage}
\author{Laureline MARTIN}
\date{Mercredi 7 Octobre 2020}

\begin{document}
\maketitle

\newpage
\renewcommand{\contentsname}{Table des matières}\tableofcontents

\newpage
\section{Introduction}
	Dans le cadre de la validation du Master 2 Data Managment in a Digital Word - Datascale, proposé par l'Université de Versailles Saint-Quentin, j'ai effectué un stage de 6 mois au sein d'un laboratoire de recherche.
	J'ai poursuivi ce stage en majorité en télétravail, dû à la crise sanitaire de la COVID-19.\par
	Dans ce rapport, je vais tout d'abord présenter le laboratoire ainsi que l'équipe avec laquelle j'ai travaillé. Puis je vais présenter ma mission principale et les tâches que j'ai effectuée.

\section{Contexte du stage}
	\subsection{Le laboratoire CEDRIC}
		Pour ce stage, j'ai été accueillie par le laboratoire de Centre d'Etudes et De Recherche en Informatique et Communication (CEDRIC)\footnote{\href{http://cedric.cnam.fr/lab/accueil/labo/}{http://cedric.cnam.fr/lab/accueil/labo/}} au sein du Conservatoire National des Arts et Métiers (CNAM) de Paris. 
		Le laboratoire se situe au 2 rue Conte 75003 PARIS.\par
		Le laboratoire CEDRIC est composé de huit équipes ayant des actions différentes dans le domaine de l'informatique fondamentale et appliquée.
		\begin{itemize}
			\item L'équipe Données complexes, apprentissage et représentations (Vertigo) : extraire de l'information et construire des méthodes de gestion de données basées sur le contenu pour des données massives audios, images et vidéos;
			\item L'équipe Interactivité pour Lire et Jouer (ILJ) : ;
			\item L'équipe Ingénieurie des Systèmes d'Information et de Communication (ISID) : ;
			\item L'équipe Traitement du signal et architectures électroniques (LAETITIA) : ;
			\item L'équipe Méthode Statistique de Data-Mining et Apprentissage (MSDMA) : ;
			\item L'équipe Optimisation Combinatoire (OC) : ;
			\item L'équipe Réseaux et Objets Connectés (ROC) : ;
			\item L'équipe Systèmes Sûrs (SYS) : ;
		\end{itemize}
	\subsection{L'équipe}
		J'ai été encadrée par 
		\begin{itemize}
			\item Mme Elena KORNYSHOVA, Maître de conférences travaillant au sein de l'équipe \textit{Igénieurie des systèmes d'information et de décision}, 
			\item Mme Viviane GAL, ingénieure travaillant au sein de l'équipe \textit{Interactivité pour lire et jouer},
			\item M Eric GRESSIER-SOUDAN, professeur des universités travaillant au sein de l'équipe \textit{Réseaux et Objets Connectés}
		\end{itemize}
		Cette équipe porte un projet exploratoire du CEDRIC. Il s'agit du projet "	".

\section{Objectif du stage}
	Le but principal de mon stage a été de pouvoir réaliser une ontologie sous forme de diagramme de classe. 
	Cette ontologie devait pouvoir mettre en évidence les briques principale pour élaborer un jeu pervasif utilisant un modèle utilisateur capable de prendre en charge les émotions du joueurs et d'autres traits liés à lui.\newline
	Pour ce faire, il a fallut tout d'abord faire l'état de l'art de ce qui existait dans la littérature scientifique.

\section{Etat de l'art}
	Le premier jour de mon stage, mes tuteurs m'ont remis 5 documents qui m'ont servi de base pour entrer dans le sujet du jeu pervasif (avec \cite{gal_2019}), la reconnaissance des émotions et des états émotionnels (avec \cite{gal_2019,gizycka_et_al._2018,nalepa_et_al._2019}), le modèle utilisateur (avec \cite{alhudar_2019}) et les Decision Making avec (avec \cite{kornyshova_et_al._2010}).\newline
	Aprèss la lecture de ces documents, 
\section{Approche suivie et solution proposée}
\section{Validation}
\section{Conclusion}


\bibliographystyle{abbrv}
\bibliography{include/biblio}
\end{document}