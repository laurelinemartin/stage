\documentclass{article}
\usepackage[utf8]{inputenc}
\usepackage[T1]{fontenc} 
\usepackage[french]{babel}
\usepackage{graphicx}
\usepackage{subcaption}
\usepackage[table]{xcolor}
\usepackage{geometry}
\usepackage{hyperref}
\usepackage{appendix}
\usepackage{pdfpages}

\geometry{hmargin=2.5cm,vmargin=3cm}
\setlength{\parskip}{0.1cm}

\title{Rapport de Stage}
\author{Laureline MARTIN}
\date{Mercredi 7 Octobre 2020}

\begin{document}
\maketitle
\vspace*{14cm}
\begin{center}
	\textbf{Encadrants :}
\end{center}
\begin{tabular}{p{4cm}p{4cm}p{8cm}}
	\hspace*{-1cm}
	Mme. Viviane GAL & Ingénieure & Equipe \textit{Interactivité pour lire et jouer - ILJ}\smallskip\\
	\hspace*{-1cm}
	M. Eric GRESSIER-SOUDAN & Professeur des universités & Equipe \textit{Réseaux et Objets Connectés - ROC}
	\smallskip\\
	\hspace*{-1cm}
	Mme. Elena KORNYSHOVA & Maître de conférences & Equipe \textit{Igénieurie des systèmes d'information et de décision - ISID}
\end{tabular}

\newpage
\begin{center}
	\textbf{Résumé :}
\end{center}
\hspace*{0.4cm}
Durant cinq mois et demi, j'ai travaillé au laboratoire de recherche CEDRIC au Conservatoire National des Arts et Métiers de Paris.\par
Ce stage s'est inscrit au sein du projet exploratoire "Conception et développement des jeux pervasifs adaptables avec la prise en compte des états émotionnels des joueurs" du laboratoire.
L'objectif global de ce projet est de proposer un jeu pervasif où le joueur et ses émotions sont au coeur même du jeu. 
Pour cela, des capteurs physiologiques sont utilisés pour détecter et reconnaître les états émotionnels du joueurs ainsi que des capteurs placés dans l'environnement pour connaître le contexte global. 
Ces capteurs génèrent beaucoup de données qu'il faut traiter en temps-réel pour une adaptation dynamique du jeu.
Dans ce rapport, nous prposons une solution pouvant ingérer les données provenant de capteurs physiologiques en temps-réel.
Ensuite, cette solution envoie ces données à des algorithmes déjà existant qui permettent de déduire des états émotionnels.
Un état de l'art ainsi qu'une première version d'une ontologie afin de répondre à cet objectif ont été proposées et se trouvent dans ce rapport. 
\bigskip\newline
\begin{center}
	\textbf{Abstract :}
\end{center}
\hspace*{0.4cm}
I worked for the CEDRIC research laboratory based on Conservatoire Nationnal des Arts et Métiers at Paris during five mounths and half.\par
This work was for the laboratory's exploratory project "Desing and development of adaptables pervasives games to players' emotional states".
The main project goal is to develop a pervasive game where player and his emotions are the heart of the game.
To reach this goal, physiological sensors are used in order to detect and recognize player's emotional states and others sensors are placed in environment to know the global context.
Sensors generate a lot of data that must be processed in real-time to a dynamic game's adaptation.
This report contains a solution to ingest real-time physiological sensors' data.
This solution sends data to existing detection and recognition of emotional states algorithms.
This report contains a state of the art and a first version of an ontology that I made too. 
\vspace*{3.6cm}
\begin{center}
	\textbf{Avant-propos}
\end{center}
\hspace*{0.4cm}
Dans le cadre de la validation du Master 2 Data Managment in a Digital World - Datascale, proposé par l'Université de Versailles Saint-Quentin / Paris Saclay, j'ai effectué un stage de cinq mois et demi du 16 Mars 2020 au 30 Septembre 2020.
Le contexte particulier lié à la crise sanitaire m'a amené à travailler pour la majorité de mon stage en télétravail (du 16 Mars au 10 Septembre).\par
Pour communiquer pendant cette période de travail à distance, nous utilisions les mails et nous avons organisé des réunions via Teams tous les 7 à 15 jours.
Ces réunions duraient entre une demie heure et une heure.
Ces sessions vidéos étaient l'occasion de faire le point sur l'avancement de mon travail et sur les tâches des jours à venir.
Cependant, la barrière de l'écran a créé un réel manque de spontanéité lors de nos échanges.
Pour ma part, je pense que les visio-conférences n'ont pas suffit pour échanger tout au long de cette période de télétravail. 
Peut-être qu'une autre forme de communication, comme une messagerie instantanée, ajoutée à celle des réunions sur Teams, aurait pu réduire se manque de spontanéité.\par
A l'issue de ce stage, je pense que les réunions en présentiel restent la meilleure manière de communiquer et de partager autour d'un projet.

\newpage
\renewcommand{\contentsname}{Table des matières}\tableofcontents

\newpage
\section{Introduction}
	Depuis plusieurs années, les jeux pervasifs se multiplient sur le marché, proposant aux joueurs des contenus très immersifs.
	Le jeu pervasif (aussi appelé jeu omniprésent) est un genre de jeu qui mêle le monde physique et le monde numérique et rend floue cette frontière.
	Dans le jeu pervasif, le contexte est essentiel. 
	Quatre facteurs sont mis en jeu : l'espace, le temps, la technologie et les rapports sociaux. 
	L'espace du jeu pervasif est à la fois ancré dans le réel et à la fois virtuel.
	Il convient alors que des ponts existent entre ces deux espaces. 
	Ces ponts peuvent se faire via des objets, des points géographiques, des personnes, etc. 
	Le temps est calqué sur le temps réel mais il peut être utilisé de différentes manières dans le jeu. 
	Par exemple, l'heure et/ou le jour peuvent influencer le comportement que doit adopter le joueur.
	La technologie est également importante. 
	Elle doit rester accessible au joueur, tant sur le plan pratique que sur le plan financier. 
	La géolocalisation, les réseaux sans fil, etc. sont des technologies utilisées pour ce type de jeu.
	Dans les jeux pervasifs, les joueurs, les spectateurs et les non-joueurs sont mélangés sur le même espace. 
	Les joueurs n'ont pas de caractéristiques particulières pour se reconnaître.%\par
	Le jeu pervasif à un un impact social et culturel important.
	Il favorise les récontres dans l'espace physique.
	Il permet de découvrir de nouveaux endroits, de nouvelles choses et de s'enrichir culturellement.
	Cependant, le jeu omniprésent ne met pas réellement le joueur au centre de jeu.\par
	Le projet "Conception et développement des jeux pervasifs adaptables avec la prise en compte des états émotionnels des joueurs" vise à mettre le joueur et ces émotions au centre du jeu.
	Ce seront les émotions du joueurs qui influenceront le jeu et ses événements.\par
	Pour pouvoir détecter et reconnaître les états émotionnels du joueur afin que le jeu s'y adapte et propose l'événement le plus adéquat, de multiples capteurs physiologiques doivent être utilisés.
	Ces capteurs généèrent beaucoup de données qui doivent être stockées et traitées.
	Ces données permettront a des algorithmes de reconnaître les états émotionnels.\par
	Le but de ce travail a été de dresser le contour et d'affiner la conceptualisation d'un jeu pervasif adaptable aux états émotionnels du joueur. 
	Dans ce rapport, nous nous sommes intéressés, d'une part, à l'adaptation de jeux selon le contexte émotionnel et d'autre part, à la gestion de données générées par des capteurs physiologiques.\par
	Dans un premier temps, nous allons présenter l'état de l'art comparant différentes approches pour l'adaptation dynamique d'un jeu pervasif aux états émotionnels du joueur. 
	Nous aborderons ensuite l'expérience que nous avons menée lors d'un escape game et la notion d'expérience utilisateur.
	Puis, nous présenterons une ontologie non-aboutie d'un jeu pervasif prenant en compte l'état émotionnel du joueur.
	Enfin, nous expliquerons une solution que nous avons conçue pour la gestion de données en temps-réel provenant de capteurs physiologiques.

\section{Présentation du laboratoire CEDRIC}
	\begin{center}
		\includegraphics[scale=0.35]{../include/logo-cedric.PNG}\\
	\end{center}
	J'ai travaillé au sein du laboratoire de Centre d'Etudes et De Recherche en Informatique et Communication (CEDRIC)\footnote{\href{http://cedric.cnam.fr/lab/accueil/labo/}{http://cedric.cnam.fr/lab/accueil/labo/}} au sein du Conservatoire National des Arts et Métiers (CNAM) de Paris\footnote{\href{http://www.cnam.fr/portail/accueil-conservatoire-national-des-arts-et-metiers-821166.kjsp}{http://www.cnam.fr/portail/accueil-conservatoire-national-des-arts-et-metiers-821166.kjsp}}. 
	Le laboratoire se situe au 2 rue Conte 75003 PARIS.\par
	Le laboratoire CEDRIC est composé de huit équipes ayant des actions dans le domaine de l'informatique fondamentale et appliquée, ainsi que dans d'autres disciplines proches (statistiques, électronique,...).
	\begin{itemize}
		\item L'équipe Données complexes, apprentissage et représentations (Vertigo) : Extraire de l'information et construire des méthodes de gestion de données basées sur le contenu pour des données massives audios, images et vidéos;
		\item L'équipe Interactivité pour Lire et Jouer (ILJ) : Questions autour de l'interaction homme-machine concernant des activités telles que le jeu et la lecteure. Cette équipe mèle plusieurs disciplines (informatiques, psychologie, design, arts,...) afin de répondre aux études en cours qui concernent la modélisation de la difficulté dans les jeux, les méthodologies de game design inclusif,...;
		\item L'équipe Ingénieries des Systèmes d'Information et de Décision (ISID) : Réunie autour de trois axes de recherche (les systèmes décisionnels, le web sémantique et la qualité des systèmes d'information) dans le but de concevoir des méthodes, des outils et des techniques	pour la conception et l'analyse de systèmes d'information et de décision dans tous les domaines;
		\item L'équipe Traitement du signal et architectures électroniques (LAETITIA) : Concentrée sur trois axes de recherhe : traitement du signal pour les télécommunications (recherches en lien avec les réseaux cellulaires (5G/6G) et en lien avec les problématiques de couches physique pour les réseaux faible puissance longue portée pour l'IoT), sûreté de fonctionnement des sytèmes dynémiques (recherche en automatique) et implémentation temps-réel (mise en oeuvre des algorithmes proposés par l'équipe);
		\item L'équipe Méthode Statistique de Data-Mining et Apprentissage (MSDMA) : Traitement de données et développement de modèles d'apprentissage statistiques dans le but d'extraire des information pour prise de décision;
		\item L'équipe Optimisation Combinatoire (OC) : Rassemblée autour de deux axes : la programmation mathématique et applications ainsi que les graphes et optimisation;
		\item L'équipe Réseaux et Objets Connectés (ROC) : Porte sur l'analyse et l'exploitation des nouvelles architectures réseaux et des systèmes liés à la virtualisation, à la mobilité et au développemnt des objets connectés;
		\item L'équipe Systèmes Sûrs (SYS) : Spécialisation, conception, vérification et évaluation des systèmes. Pour cela, Les recherches se font autour de trois axes : l'axe typage, sémantique et preuve, l'axe architecture logiciel, architecture systèmes et ligne de produit et l'axe vérification  et évaluation de systèmes parallèles et asynchrones.
	\end{itemize}

\section{Le projet}
	Le projet exploratoire intitulé "Conception et développement des jeux pervasifs adaptables avec la prise en compte des états émotionnels des joueurs" a éé initié par trois enseignants-chercheurs du laboratoire CEDRIC du CNAM de Paris.
	Il s'agit de Mme. Viviane GAL, M. Eric GRESSIER-SOUDAN et Mme. Elena KORNISHOVA.
	Ce projet vise à élaborer un jeu pervasif capable de s'adapter dynamiquement selon le contexte du joueur et son environnement.\par
	Dans cette section, nous allons présenter l'objectif global de ce projet.
	Puis, nous parlerons de la problématique à laquelle nous apporterons une solution.
	Nous aborderons aussi les sujets connexes que nous verrons tout au long de rapport.
	\subsection{Motivations et objectif du projet}
		Le but est de concevoir un jeu omniprésent capable de détecter et de reconnaître dynamiquement l'état émotionnel courant de joueur. 
		Le jeu doit aussi être capable de s'adapter à cet état en temps-réel en proposant un événement spécifique. 
		Cet événement devra soit 
		\begin{itemize}
			\item Avoir tendance à provoquer le même état émotionnel que celui expérimenté par le joueur à cet instant afin de le maintenir dans cet état émotionnel;
			\item Avoir tendance à provoquer un autre état émotionnel que celui expérimenté par le joueur à cet instant dans le but de provoquer un changement d'état émotionnel.
		\end{itemize}
		Le choix de maintenir ou de changer l'état émotionnel du joueur devra se faire selon différents critères.
		Ces critères pourraient être : le type de l'état émotionnel (positif ou négatif), l'avancement dans le jeu ou bien le contexte environnemental courant.
		Le but d'une telle adaptation est de pouvoir garantir l'expérience la plus immersive possible et un meilleur divertissement.\par
		Toutefois, la conception d'un tel jeu devra rester assez générique.
		L'une de nos motivations finale est de pouvoir appliquer notre jeu à d'autres domaines dans de prochains projets.
		D'autres domaines d'applications pourraient être le domaine du médicale ou celui de l'apprentissage.
	\subsection{Problématique}
		Pour ce projet aussi vaste qu'est le nôtre, plusieurs problématiques se posent.
		Par exemple, nous pouvons nous intéresser aux moyens de détecter de reconnaitre les états émotionnels d'une peronne.
		Nous pourrions aussi nous intéresser aux problématiques liées à l'adaptaion du jeu pervasif.
		Les méthodes et les règles qu'il faudrait utiliser pour une bonne adaptation du jeu pervasif à son joueur sont à définir.\par
		La problématique à laquelle nous avions tout d'abord tenté d'apporter une contribution était celle de la réalisation d'un modèle conceptuel pour représenter l'adaptation d'un jeu pervasifs aux émotions d'un joueur.
		Mais, au cours de notre travail, nous avons décidé d'aborder une autre problématique également importante pour l'aboutissement de notre projet.\par
		Dans ce rapport, nous allons traiter principalement la problématique de la gestion des données générées par les capteurs physiologiques.
		Pour atteindre notre objectif final, il est important de pouvoir utiliser les données des capteurs pour déterminer les états émotionnels afin que le jeu puisse s'y adapter. 
		Cela pose une contrainte importante sur la gestion des données en temps-réel.
		En effet, la réponse à l'état émotionnel courant du joueur doit être la plus immédiate possible afin de ne pas créer un sentiment de décalage entre le ressenti du joueur et l'adaptation du jeu.
	\subsection{Définitions et sujets connexes}
		Il est important de faire une mise au point sur les termes que nous allons employés tout au long du rapport.\par
		Ici, nous parlons d'état émotionnel et non d'émotion.
		L'émotion est une réaction physiologique forte qui ne dure que quelques minutes.
		Elle est la réponse à un événement, un stimuli,...
		L'ensemble des émotions est très restreint.
		Il se composent, selon les définitions de quatre à huits émotions.
		L'état émotionnel est plus vaste, il regroupe plusieurs notions en plus de l'émotion telles que le sentiment ou l'affect.
		L'état émotionnel est donc une définition "étendue" de l'émotion, ce qui nous permet de ne pas exclure des notions que nous pourrions considérer.\par
		Pour répondre à la problématique de la gestion de données provenant de capteurs en temps-réel, nous avons besoin d'avantage de contexte sur des sujets tels que : les méthodes de détections et de reconnaîssance des états émotionnels et la notion d'expérience utilisateur.
		Tout au long du rapport, nous allons aborder ces sujets qui n'entre pas directemnt dans le cadre de la solution que nous apportons mais qui sont essentiels pour la compréhension du sujet de notre projet et de l'apport de notre solution.

\section{Etat de l'art}
	Notre état de l'art s'intéresse aux différentes méthodes déjà existantes dans la littérature scientifique permettant à différents types de jeux de s'adapter aux états émotionnel de leurs joueurs.\par
	Dans l'état de l'art, nous proposons un cadre que nous avons conçu pour comparer des approches pour l'adaptation de jeux aux états émotionnels des joueurs. Le cadre s'appuie sur les critères suivants :
	\begin{itemize}
		\item Les méthodes pour la détection et pour la reconnaissance des états émotionnels
		\begin{itemize}
			\item Les capteurs utilisés
			\item Les algorithmes utlisés pour la détection et/ou la méthode de l'état émotionnel
		\end{itemize}
		\item Les autres caractéristiques de l'utilisateur prises en compte par le jeu
		\item Les méthodes d'adaptation d'un jeu au contexte de son utilisaeur
		\begin{itemize}
			\item Méthodes pour l'adaptation du jeu selon l'état émotionnel du joueur
			\item Méthodes pour l'adaptation du jeu selon d'autres caractéristiques du joueur
		\end{itemize}
	\end{itemize}
	L'état de l'art se trouve en Annexe \ref{ann:eda}.\par
	De cette comparaison, il en ressort plusieurs questions et limites.
	Tout d'abord, on remarque que seulement quelques états émotionnels sont raités dans les approches. 
	Même si on peut imaginer que certains états émotionnels ne seront jamais expérimenté au cours du jeu que nous souhaitons concevoir, il semble tout de même important de pouvoir prendre en compte le plus grand nombre d'états émotionnels possible.
	Sans quoi, nous le jeu pourrais mal s'adapter au joueur et impacter négativement son niveau d'intérêt pour le jeu.
	Dans notre cas, nous utilisons des capteurs physiologiques de contact, c'est-à-dire collé au joueur, pour aquérir des données sur son état physiologique pour déduire son état émotionnel courant via des algorithmes.\par
	On observe que certaines aproches font aussi cela en utilisants différents types d'algorithmes mais que d'autres utilises des modèles pour "prédire" l'état émotionnel courant du joueur.
	Ce genre de méthode peut être intéressante car elle retire l'équipement de capteurs du joueur.
	Cependant, ce type de méthode semble être moins précise car elle se base sur le principe que tous les joueurs ressentirons exactement le même état émotionnel lorsqu'ils se retrouveront dans le même cas de figure.
	De plus, cela implique d'envisage absolument tous les états du jeu possibles, toutes les actions possibles et d'associer un état émotionnel pou chacque intersection.
	Ce qui nous fait croire que l'utilisation de capteurs physiologiques, même si c'est plus contraignant pour le joueur, est plus pertinent.\par
	D'autres critères peuvent être considérés pour augmenter la connaissance sur le joueur.
	Ces caractéristiques peuvent être intéressantes à étudiant à la fois pour la détection et la reconnaissance de l'état émotionnel mais aussi pour l'enrichissement de la connaissance du contexte de l'environnement globale. 
	Ces caractéristiques peuvent aussi bien s'appuyer sur un questionnaire de préférence rempli par l'utilisateur que sur la présence de capteurs dans l'environnement.\par
	Cet état de l'art, nous a permis d'avoir une vision de ce qui existait déjà en matière d'adaptation de jeux aux états émotionnels des joueurs.
	Il aura aussi été l'occasion de proposer un cadre de comparaison pour ce sujet.
	Grâce à cet état de l'art, j'ai pu comprendre quels sont les enjeux pour la réalisation de notre objectif.

\section{Expérience d'un Escapte Game}
	Nous avons expérimenté sur nous-même un escape game au Victory Escape Game\footnote{\href{http://www.victoryescapegame.fr/}{http://www.victoryescapegame.fr/}} situé au 37 rue des Gravilliers à Paris.
	Cela a été l'occasion d'expérimenter ce genre de jeu très immersif.\par
	Dans cette section, nous allons tout d'abord aborder la notion d'expérience utilisateur.
	Puis nous ferons un retour sur cette expérience d'escape game et ce qu'elle a pu apporté sur le travail mené.
	\subsection{Expérience Utilisateur}
		L'expérience utilisateur (UX pour User eXperience) concrne le ressenti de l'utilisateur à chaque interaction lors de l'utilisation d'un service, d'un objet, d'un espace public, etc. 
		L'UX désigne donc un vaste domaine d'application, cependant, dans le cadre de nos recherches, nous nous intéresserons à l'expérience utilisateur d'un jeu uniquement.\par
		Lors de la conception d'un jeu, une expérience utilisateur unique ne peut pas être conçue.
		L'UX concerne le ressenti du joueur par rapport à l'utilisabilité du jeu.
		Cela signifie que l'UX est propre à chacun.
		Elle inclut les émotions des joueurs, leurs préférences, leur perception, leurs réponses physiques et psychologiques, leurs comportements, etc.
		Plusieurs méthodes pour une conception de "bonne" expérience utilisateurs existents.
		Ces méthodes varient en ne cessent de s'agrandir et de s'affiner, mais nous pouvons les répertorier comme suit :
		\begin{itemize}
			\item 
		\end{itemize}
	\subsection{Expérimentation d'un Escape Game}
		L'escape game se présentait sous la forme d'une mission d'une heure et demie.
		Le but de cette mission était d'entrer dans un vaisseau désafecté pour emprunter un portail de téléportation.
		Celui-ci étant cassé, il fallait récupérer des indices dissimulés dans les pièces pour le redémarrer.

\section{Modélisation conceptuelle}
	Dans cette section, nous allons présenter un modèle conceptuel qui n'a pas pu être terminé. 
	Le but de ce modèle conceptuel était de réprésenter chaque composant permettant la prise en compte des états émotionnels dans le jeu pervasif et de s'y adapter. 
	Il fallait à la fois représenter le joueur, le jeu, les états émotionnels, les capteurs mais aussi des composants plus abstraits comme les réactions physiologiques du joueur, la reconnaissance des états émotionnels, l'adaptation dynamique du jeu, etc. 
	Ce modèle devait être représenté sous la forme d'une ontologie en utilisant le format standard du diagramme de classes UML.
	\subsection{Ontologie}
		Le diagramme de la Figure \ref{fig:modele} représente notre modélisation pour les différents éléments qui composent un jeu pervasif adaptable aux états émotionnels du joueur. 
		\begin{figure}
			\centering
			\includegraphics[scale=0.5]{../include/ontologie_stage_cnam-v2-5.png}
			\caption{Ontologie d'un jeu pervasif prenant en compte l'état émotionnel du joueur}
			\label{fig:modele}
		\end{figure}
		Pour chaque classe, il existe une description détaillée en Annexe \ref{ann:detailclasses}
		.\par
		Dans ce modèle, on voit qu'il existe une relaion entre \texttt{Utilisateur}, \texttt{Jeu} et \texttt{Etat Emotionnel} que j'aurais pu représenter par une liaison ternaire.
		Cependadant, j'ai préféré décomposer cette association en trois relations binaires dans un soucis de clareté. 
		Une autre amélioration possible à cette représentation est celle de la relation entre \texttt{Jeu} et \texttt{Modèle Utilisateur}. 
		Avec du recul sur mon travail, je pense qu'il aurait été possible de soit fusionner les deux classes (puisqu'il s'agit d'une relation $1^* - 1$) soit de transformer la classe \texttt{Modèle Utilisateur} en une ou plusieurs autre(s) classe(s).\par
		L'ontologie que je présente dans ce rapport à la Figure \ref{fig:modele} n'est donc pas le résultat définitif.
		En effet, après plusieurs essais et plusieurs versions différentes de l'ontologie, nous avons décidé de changer de point de vue et de revenir, si possible, plus tard à l'ontologie.
	\subsection{Bilan}
		Le bilan que je tire de cet exercice de modélisation est que malgré cet échec d'aboutir à une ontologie livrable, j'ai pu d'une part mieux comprendre la problématique liée au projet et à la conception de modèle.
		Et d'autre part, j'ai pu me rendre compte de l'étendue des difficultés qu'il était possible de rencontrer dans ce genre d'exercice.
		Par exemple, lors de la conception de cette ontologie, je me suis retrouvée à mélanger plusieurs niveaux d'abstraction, ce qui la rendait peu compréhensible et invalide. 
		Je me suis aussi rendu compte qu'il fallait beaucoup de patience et d'entraînement pour concevoir un tel modèle. 
		Aujourd'hui, je comprends mieux les enjeux de tels modèles. 
		A l'avenir je pourrais réaliser des modèles de meilleure qualité grpace à cette expérience.\par
		La réalisation de cette ontologie nous a permis de réorienté notre objectif de recherche vers l'implémentation d'une solution pour la gestion des données générées par les capteurs physiologiques.

\section{Prototypage}
	Pour la détection et la reconnaissance de l'état émotionnel, nous utilisons des capteurs physiologiques de contacts.
	Ces capteurs prennent en continue des mesures pour différentes métriques physiologiques telles que la température du corps, le rythme cardiaque, la réponse électrodermale ou encore le rythme respiratoire.
	Ces capteurs génèrent beaucoup de données.
	De plus, il s'agit de données en temps-réel qui doivent être traitées le plus rapidement possible pour reconnaître l'état émotionnel de la personne et ainsi le jeu pourra s'adapter avec un minimum de la latence.
	Une trop grande latence pourrait engendrer un effet de décalage désagréable pour le joueur entre ce qu'il ressent et l'adaptation du jeu.\par
	Dans cette section, nous allons présenter le prototype de notre solution pour gérer les données temps-réel des capteurs.
	Les capteurs physiologiques et les algorithmes de reconnaîssance existent déjà, ils proviennent de précédents travaux effectués au laboratoire.
	Ici nous cherchons à gérer au mieux les données en temps-réel en créant un minimun de latence entre la mesure par le capteur et la reconnaissance de l'état émotionnel.
	% Insérer le schéma de l'utilisation du prototype par rapport à l'objectif
	\subsection{Technologie utilisée}
		Pour notre prototype nous avons besoin de gérer efficacement des données continues et en temps-réel.
		Il nous faut donc trouver une technologie capable de traiter efficacement ce genre de données pour les transmettre aux algorithmes de détection et de reconnaissance de l'état émotionnel;
		\subsubsection{Comparaison entre Apache Kafka et RabbitMQ}
			Notre choix s'est porté sur les Middleware Orientés Messages (MOM).
			Ces middleware ont pour objectif de transmettre un message d'un utilisateur à un autre.\par
			Nous avons sélectionné deux middlewares libres, simple d'utilisation et avec une communauté active.
			Il s'agit d'Apache Kafka et de RabbitMQ.
			Avan de commencer la synthèse comparative entre ces deux middlewares,  introduisons les points communs entre Apache Kafka et RabbitMQ :
			\begin{itemize}
				\item Middleware orienté message (MOM);
				\item Même ordre de grandeur de message consommable par seconde (un peu + côté Kafka);
				\item Open source;
				\item Modèle d’intégration : Publish / Subscribe (+ point-to-point côté RabbitMQ).
			\end{itemize}
			Apache Kafka et RabbitMQ sont utilisables dans les domaines d'application suivants :
			\begin{itemize}
				\item Collecte d’information pour l'Internet des Objets (IoT);
				\item Transmission de métriques;
				\item Envoi de notifications, de mails, de messages instantanées;
				\item Transmission d’actions utilisateurs;
				\item Actions compte à compte bancaire (transactions).
			\end{itemize}\par
			Maintenant que nous connaissons les points communs entre les deux MOM, nous pouvons en faire une comparaison synthétique.\par
			Dans ce comparatif, nous traitons des principales différences des deux middlewares (Tableau \ref{tab:comparatifinfos}), de leurs architectures (Tableau \ref{tab:comparatifarchi} et Figure \ref{fig:comparatifarchi}), de leurs approches (Tableau \ref{tab:comparatifapp}, des caractéristiques des messages (Tableau \ref{tab:comparatifmess}) et de la préférence d'utilisation de l'un ou de l'autre MOM selon le cas d'usage (Tableau \ref{tab:comparatifpref}).\newline
			\begin{table}
				\begin{tabular}{|p{8cm}|p{8cm}|}
					\hline
					\rowcolor{lightgray} RabbitMQ & Kafka\\\hline
					Depuis 2007 & Depuis 2011\\\hline
					Destiné au début : composante primaire dans les messageries SOA ; Maintenant : Flux & Pour le scenario de flux\\\hline
					Structure de données : FIFO; Optimal quand les messages sont livrés rapidement & Structure de données : Log (le consommateur gère l’offset) Possibilité de relire des données, les messages sont conservés selon un temps paramétré\\\hline
					Inclus les protocoles : MQTT, AMQP, STQMP; Facilité pour la communication avec d’autres solutions implémentant AMQP. & Routage simple (clé de routage). Les messages sont sur des « topics » (les consommateurs s’abonnent aux topics voulus)\\\hline
					Mode de délivrance des messages : \textit{at least once} & Modes de délivrance des messages : \textit{at least once et exactly once}\\\hline
				\end{tabular}
				\caption{Principales différences / informations générales}
				\label{tab:comparatifinfos}
			\end{table}
			\medskip
			\begin{table}
				\begin{tabular}{|p{4cm}|p{6cm}|p{6cm}|}
					\rowcolor{lightgray} & RabbitMQ & Kafka\\
					Stocakge & Dans une base Mnésia. Quand saturation sur la base Mnésia, stockage sur disque. & Sur disque, dans des fichiers (tailles équiv.), logs. Cluster de servers, dans des topics (Durable). \\\hline
					Routage & Flexible. & Basique.\\\hline
					Structure de données & File FIFO. & Log.\\\hline
				\end{tabular}
				\caption{Architectures}
				\label{tab:comparatifarchi}
			\end{table}
			\begin{figure}[ht]
				\begin{subfigure}{0.6\textwidth}
					\hspace*{-2cm}
					\includegraphics[scale=0.5]{../include/RabbitMQ.PNG}
					\caption{Schéma de l'architecture de RabbitMQ}
					\label{fig:archirabbitmq}
				\end{subfigure}
				\begin{subfigure}{0.5\textwidth}
					\includegraphics[scale=0.55]{../include/Kafka.PNG}
					\caption{Schéma de l'architecture de Kafka}
				\end{subfigure}
				\caption{Schémas des architectures}
				\label{fig:comparatifarchi}
			\end{figure}
			\medskip
			\begin{table}
				\begin{tabular}{|p{8cm}|p{8cm}|}
				\hline
				\rowcolor{lightgray} RabbitMQ & Kafka\\\hline
				\begin{center} \textbf{Approche Push-based :}\end{center} & \begin{center} \textbf{Approche Pull-based :}\end{center}\\\hline
				Distribuer les messages individuellement et rapidement. & Consommateur récupère les lots souhaités, à partir d’un offset spécifique ; Mise en commun longue.\\\hline
				\end{tabular}
				\caption{Approches}
				\label{tab:comparatifapp}
			\end{table}
			\medskip
			\begin{table}
				\begin{tabular}{|p{4cm}|p{6cm}|p{6cm}|}
					\hline
					\rowcolor{lightgray} & RabbitMQ & Kafka\\\hline
					Maintien de l'ordre & Dans une même chaîne (TCP multiplexée) & Dans une même partition\\\hline
					Temps de vie & Jusqu’à que le message soit onsommé (et retour du onsommateur reçu & Délai paramétré (ou si saturation atteinte)\\\hline
					Priorités & Possibilité de définir un degré de priorité des messages & - \\\hline
				\end{tabular}
				\caption{Structure des messages}
				\label{tab:comparatifmess}
			\end{table}
			\medskip
			\begin{table}
				\begin{tabular}{|p{8cm}|p{8cm}|}
				\hline
				\rowcolor{lightgray} RabbitMQ & Kafka\\\hline
				Besoin de routages élaborés; Utiliser des protocoles STOMP, MQTT ou AMQP...; Suivi de métriques opérationnelles & Besoin de routages simples ; Conserver (sur temps donné) et relire des messages; Mise à l’échelle; Capture d’évènement induisant un changement d’état (dans une base de données ou autre); Besoins transactionnels ; Traiter les données en parallèle.\\\hline
				\end{tabular}
				\caption{A utiliser de préférence selon le cas d'utilisation}
				\label{tab:comparatifpref}
			\end{table}
			La sitographie pour la construction de cette synthèse comparative se trouve en Annexe \ref{ann:kafkarabbitmq}.
		\subsubsection{Positionnement}
			Après mes recherches et la rédaction de la synthèse comparative, j'ai décidé d'utiliser Apache Kafka comme Middleware Orienté Message pour le prototypage de notre solution.
			Apache Kafka m'a semblé être plus indiqué dans le traitement des données venant capteurs.
			Ce middleware présente une structue pour les messages en log.
			Ce qui est plus intéressant pour notre cas qu'une structure FIFO (First In, First Out).
			En effet, la structure FIFO nous "oblige" à récupérer et à consommer les messages rapidement dans la queue des messages, alors que le log nous laisse plus de temps.
			Ce qui correspond plus à ce que l'on cherche puisque nous n'allons pas consommer toutes les données pour détecter un état émotionnel, mais seulement à certains moments (lorsqu'un changement significatif sera détecté).
			De plus, les données sont organisées dans des "topics".
			Il est donc possible de récupérer que certaines données en s'abonnant aux topics désirés.
			Cela nous sera très utilie pour de futurs travaux qui impliquerons d'autres capteurs que les capteurs physiologiques pour la reconnaissance des états émotionnels.
	\subsection{Description du prototype}
		Dans cette partie, nous allons présenter le prototype de la solution que nous avons implémenté pour répondre aux besoins de gérer des données en temps-réel.\par
		Dans un premier temps nous allons présenter le prototype et son architecture.
		Et dans un second temps, nous montrerons comment nous intégrons le prototype aux travaux déjà menés pour ce projet.\par
		\begin{figure}
			\centering
			\includegraphics[scale=0.6]{../include/schema-synt-prototype.png}
			\caption{Schéma synthétique de l'architecture du prototype}
			\label{fig:archiproto}
		\end{figure}
		La Figure \ref{fig:archiproto} montre l'architecture de notre prototype.
		Pour le moement, nous simulons les données des capteurs.
		Ces données ont été acquisent lors d'une session de jeu vidéo à partir de capteurs physiologiques (capteurs décrits dans la Section \ref{sec:capteurs}.
		Elles ont été enregistrés dans un fichier au format CSV.\par
		Les données sont lus puis envoyées au Producteur.
		Le Producteur ajoute à chaque topic la ou les données qui correspondent.
		Par exemple, les données qui corresponde au Rythme Cardiaque sont envoyés sur le topic "HR" (pour Heart Rate).\par
		Les données peuvent ensuite être consommées par un consommateur qui choisit le topic auquel il s'abonne.
		Les données sont affichés dans un terminal au fur et à mesure qu'elles sont produitent.
		Le consommateur attend toujours de nouvelles données à consommer.
	\subsection{Intégrer le prototype aux travaux déjà menés}
		Des travaux antérieurs ont été menés pour le projet globale.
		Ces tavaux concernent en particulier la détection et la reoconnaissance d'états émotionnels à partir de capteurs physiologiques.
		Cependant, les algorithmes utilisent des données préalablement acquisent.
		Autrement dit, les mesures physiologiques sont faites et enregistrées dans un fichier excel dans un premier temps.
		Et dans un second temps, les données sont traitées par les algorithmes qui ont été élaborés par des chercheurs du CEDRIC.
%		En particulier Viviane GAL dans sa thèse \cite{gal_2019}.
		Il n'y a donc pas de notion de temps-réel dans ces travaux.\par
		C'est pour quoi nous avons mis au point cette solution.
		Nous présentons dans cette partie l'intégration de notre solutions aux travaux qui ont déjà été menés.
		Pour cela, nous allons tout d'abord introduire les capteurs physiologiques qui sont utilisés par les travaux précédents.
		Puis nous allons présenter comment notre solution améliore ces travaux.
		\subsubsection{Capteurs physiologiques utilisés}\label{sec:capteurs}
			Afin que notre prototype transmette les données provenant de capteurs aux algorithmes de reconnaissance des états émotionnels, nous devons nous adapter aux capteurs déjà utilisés par le laboratoire pour les expérimentations passées et futures.
			Nous n'avons pas la possibilité de changer les capteurs, alors c'est à notre solution de s'y adpater.\par
			Les capteurs physiologiques qui sont utilisés dans le projet sont des capteurs de chez TEA Ergo\footnote{\href{https://www.teaergo.com}{https://www.teaergo.com}}.
			Ce sont les capteurs Captiv.
			Ceux que nous possédons permettent la mesure de la fréquence respiratoire, du rythme cardiaque, de la température (à la surface de la peau) et de la réponse électrodermale.
		\subsubsection{Schéma de l'intégration du prototypes aux travaux déjà menés}
			\begin{figure}
				\centering
				\includegraphics[scale=0.5]{../include/schema-global.png}
				\caption{Schéma de l'architecture du prototype intégré au projet}
				\label{fig:archiglobale}
			\end{figure}
			L'architecture présentée à la Figure \ref{fig:archiglobale} nous permet de voir l'intégration de notre solution aux travaux menés précédemment.
			Comme il existe déjà un algorithme qui permet de lire les données des capteurs et de les afficher sur un terminal, nous l'utilisions.
			Le Connecteur que nous utilisons permet de récupérer des données sur un terminal et de les transmettre au Producteur sous la bonne forme pour être envoyées dans les topics.
			Les topics sont ensuite consommés par les algorithmes de reconnaissance de l'état émotionnel.
	\subsection{Validation}
		L'objectif que nous avions pour ce prototype était de pouvoir récupérer des données provenant de capteurs et de les envoyés vers les algorithmes de détection et de reconnaissance d'états émotionnels\par
		Dans ce rapport, nous prposons une solutiion que nous avons implémenté.
		Elle est actuellement testée avec un fichier au format .CSV contenant les données de quatre capteurs physiologiques : la fréquence respiratoire, la réponse électrodermale, la température corporelle et la fréquence cardiaque.
		Chaque ligne de ce fichier est envoyée dans les topics avec 30 milisecondes (ms) d'écart.
		Ces 30ms correspondent au temps moyen d'acquisition entre deux données.
		Cela nous donne une impression de temps-réel.
		Les données sont ensuite lues par un consommateur qui les affiche au fur et à mesure dans un terminal.\par
		Du point de vue de l'objectif que nous avions fixé pour ce prototype, nous pouvons dire que cette implémentation correspond à ce que nous voulions réaliser.
		Les données sont bien lues et renvoyées pour être analyses.
			
\section{Conclusion}
	L'objectif principal de ce projet est réaliser un jeu pervasif adaptable dynamiquement aux états émotionnels du joueur.
	Pour ce faire, il nous utilisons des capteurs physiologiques qui génèrent des données analysées par des algorithmes pour la détection et la reconnaissance e l'état émotionnel.
	Une fois l'état émotionnel reconnu, le jeu peut alors proposer au joueur l'événement le plus adéquat pour garantir le meilleur divertissement possible aux joueurs.
	Avec cet objectif en tête, nous avons chercher un moyen efficace de gérer les données fournies en continu par les capteurs.\par
	Dans ce rapport, nous avons présenté plusieurs travaux qui ont été mené tout au long de ce travail.
	Tout d'abord nous avons fait l'état de l'art des méthodes permettant la détection et la reconnaissance d'états émotionnels dans le domaines des jeux.
	Puis, nous avons aborder la notion d'expérience utilisateur et l'expérimentation que nous avons mené lors d'un escape game.
	Ensuite, nous avons présenté notre première tentative de solution sous la forme d'un modèle conceptuel pour représenter l'adaptation d'un jeu pervasif aux états émotionnel d'un joueur.
	Enfin nous avons présenté le prototype de notre solution pour la gestion des données provenant de capteurs physiologiques pour que ces données soient utilisées avec des algorithmes déjà existants.\par
	Cette dernière solution que nous avons détaillée dans ce rapport a été implémenté permet de lire et d'envoyer en temps-réel (avec le minimum de latence possible) les données sur des topics.
	Durant ce travail de près de six mois, j'ai beaucoup appris.
	J'ai appris une certaine rigueur.
	Une rigueur tout d'abord dans mes lectures.
	Au fil des lectures que j'ai faites, j'ai noté de plus en plus de détails dans mes résumés pour éviter d'avoir à y revenir trop souvent,
	Une rigueur au niveau de la rédaction.
	J'ai appris a être très rigoureuse sur la forme de mes rédactions.\par
	Ce travail a été l'occasion d'affiner les contour du projet.
	Plusieurs perspectives sont a envisager.
	Tout d'abord il faudrait mettre à l'épreuve notre prototype directement avec nos capteurs.
	Si les tests avec les capteurs sont satisfaisants, alors il sera possible de passer à l'étape de la conception d'algorithmes pour l'adaptation selon l'état émotionnel courant du joueur.
	Il faudra par la suite développer d'autres aspects lié au jeu lui-même.
	Comme par exemple la jouabilité ou le game design.
	Un objectif de ce projet est d'utiliser ce jeu à d'autres domaines d'applications que celui du divertissement.
	On peut imaginer utiliser ce jeu pour enseigner des émotions par exemple.
	Il faudra donc garder un aspect très générique de ce jeu.


\newpage
\appendix
\section{Adaptation d’un jeu pervasif particularisé basée sur l'état émotionnel et sur les caractéristiques du joueur – État de l’art}\label{ann:eda}
	\includepdf[pages=-]{../include/eda.pdf}

\section{Description détaillée des classes de l'ontologie}\label{ann:detailclasses}
	\includepdf[pages=-]{../include/descriptionClassesOntologie.pdf}

\section{Sitographie pour la construction de la synthèse comparative entre Kafka et RabbitMQ}\label{ann:kafkarabbitmq}
	\hspace*{-0.5cm}\href{https://www.upsolver.com/blog/kafka-versus-rabbitmq-architecture-performance-use-
	case}{https://www.upsolver.com/blog/kafka-versus-rabbitmq-architecture-performance-use-
	case} \textit{[en]} \medskip\newline
	\href{https://blog.ippon.fr/2018/03/27/comparatif-rabbitmq-kafka/}{https://blog.ippon.fr/2018/03/27/comparatif-rabbitmq-kafka/} \textit{[fr]}\medskip\newline
	\href{https://www.cloudamqp.com/blog/2015-05-18-part1-rabbitmq-for-beginners-what-is-rabbitmq.htm}{https://www.cloudamqp.com/blog/2015-05-18-part1-rabbitmq-for-beginners-what-is-rabbitmq.htm} \textit{[en]}\medskip\newline
	\href{https://openclassrooms.com/fr/courses/4451251-gerez-des-flux-de-donnees-temps-reel/4451521-metamorphosez-vos-applications-temps-reel-avec-kafka}{https://openclassrooms.com/fr/courses/4451251-gerez-des-flux-de-donnees-temps-reel/4451521-metamorphosez-vos-applications-temps-reel-avec-kafka} \textit{[en]}
	%\includepdf[pages=-]{../include/comparatifKafkaVSRabbitMQ.pdf}	

\section{Offre de stage}\label{app:annexe1}
	\textbf{Elaboration du modèle conceptuel des jeux pervasifs adaptables avec la prise en compte des états émotionnels des joueurs}\par
	\medskip
	\textbf{Contexte :}\newline
	Le champ des jeux affectifs est nouveau. Il s’appuie sur l’intégration de nouveaux moyens à développer dans les jeux afin d’adaptabilité. [1] et [2] présentent une méthodologie unifiée pour la conception des jeux affectifs utilisant le plus tôt possible le mécanisme de boucle émotionnelle. Ils repèrent des variations à l’aide de mesures physiologiques et appliquent un modèle issu d’un ensemble construit considéré comme en relation avec les émotions. Leur étude montre combien la dimension émotionnelle de l’utilisateur est importante mais difficile à gérer.\newline
	Le profil du joueur, y compris ses émotions, impacte la conception des jeux. Afin de proposer une meilleure expérience aux joueurs et de proposer un jeu particularisé, le jeu doit être adaptable en fonction du contexte global du joueur. Nous sommes dans une approche holistique qui combine à la fois l’individu et ses émotions, et, les influences de l’entourage qui va du bâtiment lui-même à l’atmosphère que dégage le lieu. Très peu de travaux ont été faits pour la conception et le développement des jeux adaptables dynamiquement. [3] formalise le concept des jeux appliqués aux visites de musées. Ce travail modélise le jeu de visite et propose un processus d’équilibrage entre la dimension ludique et la dimension non ludique (la visite) de ce type de jeux. [3] propose des patrons de mission qui servent d’éléments réutilisables lors de la conception des jeux, mais qui ne couvrent qu’une partie du processus de conception.\par
	\textbf{Sujet :}\newline
	Il s’agit dans ce stage d’élaborer un modèle conceptuel du jeu pervasif adaptable basé sur les émotions. Ce modèle, éventuellement réalisé sous forme d’une ontologie, doit couvrir toute la variété des facteurs qui impactent le jeu tels que le profil de l’utilisateur et ses données physiologiques exprimant son état émotionnel. Cette ontologie doit être construite de façon à ce qu’elle soit adaptée à la démarche situationnelle nécessaire pour la composition dynamique du jeu.

\bibliographystyle{abbrv}
\bibliography{../include/biblio}
\end{document}